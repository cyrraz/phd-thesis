%====================================================================================================
\chapter{Introduction} \label{ch:introduction}
%====================================================================================================
The standard model of particle physics (\SM) is a theoretical framework describing the known elementary particles and their interactions, and was mainly developed in the second half of the last century \cite{Glashow:1961tr,Weinberg:1967tq,Salam:1968rm,tHooft:1972tcz}.
Since then, most of the predictions of the \SM were successfully verified by a large number of experiments around the world.
One of the most famous results is the discovery in 2012 of the Higgs boson by the ATLAS and the CMS experiments \cite{ATLAS:2012yve,CMS:2012qbp}, nearly 50 years after the existence of this particle had been predicted from theoretical considerations \cite{Englert:1964et,Higgs:1964pj,Guralnik:1964eu}.

Despite its experimental success, the \SM is not the theory of everything.
In particular, the \SM does not include a theory of gravity and is currently unable to explain the origin of dark matter, an unknown type of matter whose presence is deduced from gravitational effects \cite{Zwicky:1937zza,Clowe:2006eq,Planck:2018vyg}.
Moreover, several experimental results appear to contradict some predictions of the \SM, like the measured anomalous magnetic moment of the muon \cite{Muong-2:2021ojo,Muong-2:2006rrc}, or the evidence for lepton flavour non-universality \cite{LHCb:2017avl,LHCb:2020lmf,LHCb:2020gog,LHCb:2021trn}.

These experimental tensions with the \SM point to the presence of new physics (\NP), not described by the \SM.
A large number of extensions of the \SM were proposed in the last decades, predicting the existence of new particles or new interactions, and an important role of experimental particle physics is to exclude or at least constrain the proposed models, by measuring observables that are sensitive to \NP.

The goal of this thesis is to search for the decay of a $B$ meson into a $K$ meson and a pair of neutrinos, noted \BKnn.
This decay has never been observed, it is predicted with accuracy in the \SM, and, as will be shown, is an excellent probe of physics beyond the \SM.
To achieve this goal, data collected by the Belle II detector \cite{Abe:2010gxa} are analysed.
This detector is located along the SuperKEKB accelerator \cite{AKAI2018188}, an electron-positron collider in Tsukuba, Japan, working at an energy just above the threshold to produce pairs of $B$ mesons.

This thesis is organised as follows:
\bi
\item \cref{ch:theory} presents the theoretical motivation to search for \BKnn decays.
It starts with an overview of elements of the \SM that are important to describe these decays, followed by an explanation of why the decays of $B$ mesons are important tests of the \SM, a theoretical computation of the \BKnn decay probability, and a summary of previous experimental results.
\item \cref{ch:setup} describes the experimental setup, namely the SuperKEKB accelerator, needed to produce $B$ mesons, and the Belle II detector, designed to observe the decay products of $B$ mesons.
\item \cref{ch:data_analysis} introduces several important data analysis techniques and tools that are used and referred to in \cref{ch:search_for_b2knn} to select \BKnn decay candidates, and to measure the decay probability. 
\item \cref{ch:search_for_b2knn} presents every step of the method that is developed to select \BKnn decay candidates, how data compares to simulation, what are the sources of systematic uncertainties, and finally the obtained results.
\ei

If you read this document in a portable document format (\pdf), note that even if not highlighted, the references to chapters, sections, figures, tables, bibliography, pages, as well as most acronyms (including \pdf), are clickable links, following an idea proposed by Tim Berners-Lee \cite{www}.
