%====================================================================================================
\clearpage
\section{Post-fit normalisation parameters} \label{sec:pulls}
%====================================================================================================
This appendix gives information about the normalisation nuisance parameters obtained after a fit of the binned-likelihood model to data samples of \lumionpartial collected at the \Y4S resonance and \lumioffpartial collected 60\mev below the resonance (\cref{sec:first_iteration}).

If $s$ is a background sample taken from the set of considered background sources,
\begin{equation*}
s\in\left\{\epem\to\BpBm,\epem\to\BzBzb\right\}\cup\left\{\epem\to\qqbar\,:\,q=u,d,c,s\right\}\cup\left\{\epem\to\tautau\right\},
\end{equation*}
then the background normalisation shift of $s$ is defined as $\mu_{\mathrm{s}}-1$,
where $\mu_{\mathrm{s}}$ is the background strength (i.e.~the background normalisation nuisance parameter) of the background sample $s$ (\cref{sec:syst_norm}).
For example, a normalisation shift of zero corresponds to no variation with respect to the expectation, and a shift of $+0.5$ corresponds to an upscale of the expected yield by a factor of 1.5.

\cref{fig:63_fit_pulls} (top) presents the post-fit normalisation shifts of the background sources.
The normalisation shifts of the \B meson background are close to zero (no variation).
By contrast, the normalisation shifts of the $\epem\to\ccbar$ background and the $\epem\to\ssbar$ background are close to 0.4, meaning that the corresponding post-fit yields are upscaled by a factor of 1.4 with respect to the pre-fit expectations.
This upscaling of the continuum background was anticipated by the normalisation discrepancy observed when comparing simulated continuum background and off-resonance data in the bins of the signal search region at the end of \cref{sec:bdtc}.

\cref{fig:63_fit_pulls} (bottom) shows the correlation between the signal strength $\mu$, defined at the beginning of \cref{sec:da_pyhf}, and the strengths $\mu_s$ of the seven background sources.
The signal strength is anti-correlated to the strength of the charged \B meson background, because this background is similar to signal in the most sensitive bins of the signal search region (\cref{fig:63_post_fit} in \cref{sec:first_iteration}).
In addition, the strength of the charged \B meson background is strongly anti-correlated to the strength of the neutral \B meson background, meaning that these two background sources are hard to distinguish from each other.
Similarly, the strengths of the two main sources of continuum background, $\epem\to\ccbar$ and $\epem\to\ssbar$, are strongly anti-correlated.

\figs{63_fit_pulls}
{0.65}
{figs/search_for_b2hnn/first_iteration/bg_shift.pdf}
{0.85}
{figs/search_for_b2hnn/first_iteration/corr_data_fit.pdf}
{
At the top, post-fit normalisation shifts and uncertainties of the background sources as obtained by the \sghf fit to the data.
The normalisation shift is defined as $\mu_{\mathrm{bkg}}-1$, where $\mu_{\mathrm{bkg}}$ is the background strength (i.e.~the background normalisation nuisance parameter).
At the bottom, correlation between the signal strength $\mu$ and the strengths of the seven background sources.
The strength of the neutral $B$ meson background is noted $\mu_{\mathrm{mxd}}$, and the strength of the charged $B$ meson background is noted $\mu_{\mathrm{chg}}$.
}
