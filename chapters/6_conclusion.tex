%====================================================================================================
\chapter{Conclusion} \label{ch:conclusion}
%====================================================================================================
This thesis has documented a search for \BKnn decays at the Belle II experiment with an inclusive tagging method.

\cref{ch:theory} has shown how the standard model of particle physics predicts the branching fraction of this decay via an effective Hamiltonian.
An experimental determination of this branching fraction is not only an important test of the theory.
It is also sensitive to a potential violation of the lepton flavour universality, because all the neutrino flavours contribute to \BKnn, and it imposes constraints on several new physics models predicting the existence of invisible particles, since a search for \BKnn is also a search for $\B\to K+\,\mathrm{invisible}$.

\cref{ch:setup} has given an overview of the employed experimental setup, namely the SuperKEKB accelerator, which produces pairs of $B$ mesons by colliding electrons and positrons at the energy of the \Y4S resonance, and the Belle II detector, which is designed to detect the decay products of $B$ mesons.

\cref{ch:data_analysis} has introduced several analysis tools and techniques, and in particular, boosted decision tree classifiers and input variables that are function of the momentum distribution in an event.
These variables are central in the selection of \BKnn decays, because an event containing a \BKnn decay tends to exhibit a large missing momentum due to the undetected neutrino pair.

\cref{ch:search_for_b2knn} has detailed each step of the selection of \BKpnn and \BKznn decays in a data sample of \lumion collected at the \Y4S resonance and a complementary sample of \lumioff collected 60\mev below the resonance.
A selection strategy that had never been employed to search for these decays, the inclusive tagging method, is defined.
The procedure is validated by comparing data and simulation in several regions of the phase space, and by checking that the maximum-likelihood fit procedure provides an unbiased measurement of the branching fraction of \BKnn.
For this amount of data, the expected upper limits on the branching fraction of \BKpnn and \BKznn are determined from simulation to be \limitKp and \limitKz at the 90\% confidence level, respectively.

In a first iteration of the method, restricted to data samples of \lumionpartial collected at the \Y4S resonance and \lumioffpartial collected 60\mev below the resonance, no signal is observed, and an upper limit on the branching fraction of \BKpnn is determined to be \limitKppartial at the 90\% confidence level.
This result is published in \cite{Belle-II:2021rof}.
If translated into an uncertainty on the branching fraction,  the inclusive tagging method is a factor of 3.5 better per integrated luminosity than the hadronic tagging of \cite{Belle:2013tnz}, approximately 20\% better than the semileptonic tagging of \cite{Belle:2017oht}, and approximately 10\% better than the combined hadronic and semileptonic tagging of \cite{BaBar:2013npw}.

The success of the inclusive tagging method opens new opportunities for the study of rare decays involving neutrinos in the final state.
Examples of such decays include $\B\to\kaon^*\nu\bar{\nu}$, $\B\to\kaon\tau^+\tau^-$, and $\B\to\kaon\jpsi(\to\nunub)$.
Method improvements are also possible.
By combining different types of classifiers (decision trees, neutral networks) or different types of \B meson tagging (hadronic, semileptonic, inclusive), the overall sensitivity may increase.
With the new data that the Belle II and the LHCb experiments are planning to collect in the next decade, long-awaited results in flavour physics are expected in a near future.

On a broader perspective, the success of the standard model in predicting a rich variety of experimental results is undeniable.
Yet, by looking at the history of science, it is very likely that most of our current understanding of the universe in general, and of particle physics in particular, is wrong.
When feeling satisfied to see symmetries in the theory, are we so different from Kepler, who was postulating that the planet orbits were contained in an ideal succession of platonic solids?
If we are not to trust too much our interpretations, we can at least believe in the results of experiments.
To quote Cicero in \textit{De natura deorum}, ``Time erases the fictions of opinion, but confirms the judgments of nature \cite{cicero}.''
