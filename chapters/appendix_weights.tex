%====================================================================================================
\clearpage
\section{Correction weights for simulated events} \label{sec:weights}
%====================================================================================================

\cref{tab:weights} summarises the weights that are applied to simulated events to correct for multiple sources of mis-modelling:
\bi
\item The selection efficiency of the \PID requirement imposed on signal \Kp candidates (\cref{sec:candidate_selection}) differs between data and simulation.
The Belle II performance group provides weights, noted $w_{\mathrm{PID}}$, that correct for this efficiency difference.
These weights are functions of the transverse momentum $p_T$ and the polar angle $\theta$ of the signal \Kp candidate.
\item By default, simulation does not take into account the \qq-dependence of the form factor entering in the computation of the \BKnn branching fraction (\cref{sec:branching_ratio}).
A correction weight, noted $w_{ff}$, is applied to the simulated signal events to produce a realistic \qq-dependence of the signal branching fraction.
This weight is computed as the ratio between the two distributions shown in \cref{fig:form_factor} (\cref{sec:branching_ratio}).
\item In \cref{sec:bdtc}, a correction weight is introduced to correct for the mis-modelling of the continuum background simulation.
This weight is noted $w_{c}$ and is function of the \bdtc output.
\ei
The total correction weight applied to a simulated event is the product of the individual correction weights.
\vspace{1cm}

\begin{table}[hb]
\caption{
The product of a set of correction weights is applied to each simulated event.
The weights in the table are defined for the selection of \BKpnn decays.
For the neutral mode (\BKznn), the \PID weight $w_{\mathrm{PID}}$ is absent, because there is no \PID requirement (\cref{sec:candidate_selection}), and the other weights are the same as for the charged mode.
}
\begin{center}
\begin{tabular}{ll}
\input{tables/weights.tex}
\end{tabular}
\end{center}
\label{tab:weights}
\end{table}
