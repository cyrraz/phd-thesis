%====================================================================================================
\begin{abstractpage}{Abstract}
This thesis documents a search for the rare decay of a $B$ meson into a $K$ meson and a pair of neutrinos at the Belle II experiment, which is located along the SuperKEKB energy-asymmetric electron-positron collider.
This decay has never been observed, its branching fraction is predicted with accuracy in the standard model of particle physics, and is a good probe of physics beyond the standard model.
A novel method to search for this decay, the inclusive tagging, is developed on a data sample corresponding to an integrated luminosity of \lumion collected at the \Y4S resonance, and a complementary sample of \lumioff collected 60\mev below the resonance.
For this integrated luminosity, the expected upper limits on the branching fraction of \BKpnn and \BKznn are determined from simulation to be \limitKp and \limitKz at the 90\% confidence level, respectively.
When the method is applied to data samples of \lumionpartial collected at the \Y4S resonance and \lumioffpartial collected 60\mev below the resonance, no significant signal is observed, and an upper limit on the branching fraction of \BKpnn is determined to be \limitKppartial at the 90\% confidence level.
\end{abstractpage}
%====================================================================================================
\begin{abstractpage}{Zusammenfassung}
Diese Arbeit dokumentiert die Suche nach dem seltenen Zerfall eines B-Mesons in ein K-Meson und ein Paar Neutrinos mit dem Belle II-Experiment, am Elektron-Positron-Beschleuniger SuperKEKB.
Dieser Zerfall eines B-Mesons wurde bis dato noch nie direkt beobachtet, sein Verzweigungsverhältnis ist im Standardmodell der Teilchenphysik jedoch genau vorhergesagt. Er stellt somit einen exzellenten Kandidaten dafür dar, mögliche Physik jenseits des Standardmodells zu untersuchen.
Eine neue Methode zur Suche nach diesem Zerfall, das „Inclusive Tagging“, wird entwickelt auf einen Datensatz von \lumion integrierter Luminosität, aufgezeichnet bei einer Schwerpunktsenergie, die der \Y4S-Resonanz entsoricht, ergänzt durch einen Datensatz von \lumioff, aufgezeichnet 60\mev unterhalb der Resonanz.
Basierend auf der Menge dieser Daten ist eine obere Grenze auf das Verzweigungsverhältnis von bis zu \limitKp für \BKpnn und \limitKz für \BKznn zu erwarten bei einem Konfidenzniveau von 90\%.
Angewendet auf einem Datensatz von \lumionpartial, genommen an der \Y4S-Resonanz und \lumioffpartial unterhalb der Resonanz wird kein signifikantes Signal beobachtet.
Damit lässt sich ein Verzweigungsverhältnis von bis zu \limitKppartial für \BKpnn Zerfälle ausschließen bei einem Konfidenzniveau von 90\%.
\end{abstractpage}
%====================================================================================================
