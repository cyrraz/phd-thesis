%====================================================================================================
\clearpage
\section{Yield stability in the signal search region} \label{sec:yield_stability}
%====================================================================================================
The Belle II dataset is segmented into several experiments, each corresponding to a certain data collection period.
Moreover, each experiment is subdivided into runs, each corresponding to an uninterrupted period of data taking.
In this way, each event at Belle II is unambiguously defined by the triplet (experiment number, run number, event number).
The amount of on-resonance data studied here corresponds to a total of \lumion of integrated luminosity, with experiment numbers ranging from 7 to 18.

\cref{fig:yield_stability} shows the number of on-resonance data events in the signal search region per unit of integrated luminosity as a function of the data taking period for the charged mode (\BKpnn) and the neutral mode (\BKznn).
In order to follow the closed-box principle mentioned at the beginning of \cref{sec:data_mc}, the average number of events is subtracted to hide the actual number of selected data events in the signal search region.
The data yield per unit of integrated luminosity is reasonably stable, with fluctuations that may be due to the fact that the simulated events are run-independent, meaning that they are calibrated globally for the entire data sample.

\figs{yield_stability}
{1.0}
{figs/search_for_b2hnn/Bplus2Kplus_v34_yield_stability_signal_search_region.pdf}
{1.0}
{figs/search_for_b2hnn/Bzero2Kshort_v34_yield_stability_signal_search_region.pdf}
{
Number of on-resonance data events in the signal search region in bunches of 1\invfb as a function of the experiment number and the run number.
The average number of events is subtracted to hide the actual number of selected data events.
The eight top plots correspond to the \BKpnn mode, and the eight bottom plots correspond to the \BKznn mode.
}
